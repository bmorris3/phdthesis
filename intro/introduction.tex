
%We're finding planets at an ever increasing rate.

When my parents gave me my first books on astronomy, the glossy pages were filled with images from the Viking, Galileo, and Voyager missions, cataloging the worlds within the solar system with fantastic color images. Everything there was to know about planets was wrapped up neatly into books about those (then) nine planets. At the same time -- eleven months before I began my formal education in the form of kindergarten -- the first planet orbiting a star like the Sun was discovered, orbiting 51 Pegasi \citep{Mayor1995}. This new world, called an exoplanet -- or a planet that orbit a star other than the Sun�?�would be the first of many. In the following years, the number of known planets would rise at an exponential rate through to the present. There has never been a year in my age in years exceeded the number of known planets, and never will be.

%We can characterize exoplanets with following techniques.

Exoplanets have been discovered principally via the transit method. This technique requires precise measurements of stellar brightness, or photometry, to discover worlds. The arcs of some exoplanet orbits bring the worlds between their host stars and the Earth -- allowing us to see an eclipse of the host star by the exoplanet, an event known as a transit. Precision photometry can reveal the small decrease in apparent brightness of the host star. 

Transit events transmit valuable information about exoplanets. The fraction of light missing during a transit event is simply the ratio of cross-sectional areas of the planet and star, $\Delta F/F \approx (R_p/R_\star)^2$, revealing the ratio of the planetary and stellar radii. For example, an Earth-sized planet orbiting a Sun-sized star only obscures 0.008\% of the brightness of the star. The time between transit events is the orbital period of the planet, which is related to the distance between the planet and its host star, or the orbital semi-major axis, via Kepler's third law. With the orbital separation and a bit of knowledge about the host star's luminosity, we can make a crude estimate of the equilibrium temperature of the planet -- and thus whether or not it might be a world suitable for life.

%But we only know as much about planets as we know about their host stars. 

We are already discovering that we can only know�as much about a planet as we know about its host star. The planet's radius is measured relative to the stellar radius, and the planet's temperature is estimated relative to the stellar luminosity (and other terms). In order to make robust statements about exoplanets and their habitability, a few things must be known about their stars first.  

%Here's how we generally know about exoplanet host stars. 
Customarily, exoplanet host stars have been given cursory characterization upon discovery�of their planets. This basic reconnaissance might include the use of broad-band photometry (color) to measure the temperature of the star, and parallactic distances to measure the stellar luminosity when available, in combination with�expensive high resolution spectroscopy to identify the evolutionary state of the star. These measurements give an estimate of the stellar mass and radius, allowing us to estimate the absolute radius of the planet -- which is often used to confirm its nature as a planetary body, rather than a larger, stellar object -- and its surface temperature. 

For a subset of these planets, we can also learn their masses using radial velocity measurements of the host star. If the planet is massive enough or close enough to its host star, and the host star isn't particularly active (more on this later), the gravitational acceleration of the star due to the planet is detectable via spectroscopy by looking for small Doppler shifts in the�stellar spectrum. The acceleration of the star is proportional to the distance and mass of the exoplanet. If the planet is a transiting exoplanet, we know the orbital distance already via Kepler's third law, by knowing the planet's orbital period and the host star's mass, and thus we get a direct measurement of the mass of the planet. Combining the mass and radius measurements, we can get bulk density measurements of an exoplanet, indicating perhaps whether it is rocky or gaseous. 

%Stars are active. Here's what stellar activity is.
Thus far our triumphant narrative has glided over some increasingly important details. One planet-hosting star has been studied for thousands of years -- the Sun -- and millennia of observations show that the Sun bears dark (cool) spots on its rotating surface. 

% Some of the most interesting planet-hosts are active.

% We can study stellar magnetic activity like this.

% Stellar activity affects exoplanet characterization in the following ways