
%We're finding planets at an ever increasing rate.
When my parents gave me my first books on astronomy, the glossy pages were filled with fantastic color images from the Viking, Galileo, and Voyager missions, cataloging all of the known worlds�beyond Earth. Everything there was to know about planets was wrapped up neatly into books about those (then nine) planets. At the same time -- eleven months before I began kindergarten -- the first planet orbiting Sun-like was discovered, 51 Pegasi b \citep{Mayor1995}. This new world, called an exoplanet -- or a planet that orbit a star other than the Sun�--�would be the first of many. In the following years, the number of known planets would rise at an exponential rate through to the present. I am a member of the first generation for whom there has never been a time in our lives when our age in years has exceeded the number of known planets.

%We can characterize exoplanets with following techniques.
Exoplanets have been discovered principally via the transit method. This technique requires precise measurements of stellar brightness, or photometry, to discover worlds. The arcs of some exoplanet orbits bring the worlds between their host stars and the Earth -- allowing us to see an eclipse of the host star by the exoplanet, an event known as a transit. Precision photometry can reveal the small decrease in apparent brightness of the host star during the eclipse. 

Transit events transmit valuable information about exoplanets. The fraction of light missing during a transit event is simply the ratio of cross-sectional areas of the planet and star, $\Delta F/F \approx (R_p/R_\star)^2$, revealing the ratio of the planetary and stellar radii. For example, an Earth-sized planet orbiting a Sun-sized star only obscures 0.008\% of the brightness of the star. The time between transit events is the orbital period of the planet, which is related to the distance between the planet and its host star, or the orbital semi-major axis, via Kepler's third law. With the orbital separation and a bit of knowledge about the host star's luminosity, we can make a crude estimate of the equilibrium temperature of the planet -- and thus whether or not it might be a world suitable for life.

%But we only know as much about planets as we know about their host stars. 
In short order, we have already discovered that we can only know�as much about a planet as we know about its host star. The planet's radius is measured relative to the stellar radius, and the planet's temperature is estimated relative to the stellar luminosity (and other terms). In order to make robust statements about exoplanets and their habitability, we must know a few things about their stars first.  

%Here's how we generally know about exoplanet host stars. 
Customarily, exoplanet hosting stars are given cursory characterization upon discovery�of their planets. This basic reconnaissance might include the use of broad-band photometry (color) to estimate the temperature of the star, and parallactic distances to measure the stellar luminosity when available, in combination with�expensive high resolution spectroscopy to identify the evolutionary state of the star. These measurements give an estimate of the stellar mass and radius, allowing us to estimate the absolute radius of the planet -- which is often used to confirm its nature as a planetary body, rather than a larger object like a brown dwarf or small star -- and its surface temperature. 

For a subset of these planets, we can also learn their masses using radial velocity measurements of the host star. If the planet is massive enough or close enough to its host star, and the host star isn't particularly active (more on this later), the gravitational acceleration of the star due to the planet is detectable via spectroscopy by looking for small Doppler shifts in the�stellar spectrum. The acceleration of the star is related to the distance and mass of the exoplanet. If the planet is a transiting exoplanet, we know the orbital distance already via Kepler's third law, by knowing the planet's orbital period and the host star's mass, and thus we get a direct measurement of the mass of the planet. Combining the mass and radius measurements, we can get bulk density measurements of an exoplanet, indicating perhaps whether it is rocky or gaseous. 

% We can measure masses with TTVs
In transiting multi-planet systems, we can measure the masses of exoplanets without expensive spectroscopy by using transit timing variations (TTVs) \citep{Agol2005,Holman2005}. The orbit of a single transiting planet around a single star would be perfectly periodic (barring relativistic effects), but if there's more than one planet in the system, the gravitational influence of each planet on each other pulls the planets slightly ahead or behind in their orbits. The apparent early or late arrival of an exoplanet transit thus transmits information about the mass of the perturbing planet. This technique is especially promising for measuring the masses of small planets in the habitable zones of their Sun-like host stars -- which are typically too far and have too little mass to impart measurable radial velocities on their host stars with current high resolution spectroscopy. 

% We can measure atmospheric composition with transmission spectroscopy
Transits can also reveal the composition of exoplanet atmospheres via transmission spectroscopy. When the planet passes in front of the host star, the planet will appear largest at wavelengths where the planet's atmosphere is opaque, and smallest at wavelengths where the atmosphere is transparent. Thus by measuring the apparent radius of the planet as a function of wavelength, we can attain a spectrum of an exoplanet's atmosphere. Encoded in this spectrum are absorption features from the atoms and molecules in the planet's upper atmosphere, yielding clues to the atmospheric composition, the�state of the planet's climate, and perhaps even whiffs of biosignatures, or signs of life.

%Stars are active. Here's what stellar activity is.
Thus far, our triumphant narrative of planet characterization has glided over some increasingly important details about the stars, which we have taken for granted as simple, uniform disks. One planet-hosting star has been studied for thousands of years: the Sun. Even before the invention of the telescope it was known that the Sun bares dark spots on its surface \citep{Hayakawa2017}. Sunspots are regions in the solar photosphere where strong magnetic fields inhibit convection of hot plasma from below, inhibiting heat transport and creating cool, dark spots on the Sun's surface \citep{Solanki2003}. As the Sun rotates, sunspots roll into and out of view, subtly changing the light we receive from the Sun as a function of wavelength and time.

% Planet-hosting are active.
Since the time of Copernicus, astronomical advances consistently remind us that the Earth is not at the center of the Universe, and that there is nothing particularly special about our location \citep{Copernicus1543} -- so perhaps it should not come as a surprise to learn that the Sun and Solar System are not unique. Multi-planet systems like ours are common \citep{Dressing2013,Burke2015,Coughlin2016}. The processes that drive magnetic activity on the Sun, or analogous ones for smaller stars, are also ubiquitous among planet-hosting stars. The evidence for those starspots comes to us from photometry -- the same technique used to discover planets -- revealing the changes in brightness as spotted planet-hosting stars rotate, for example, with photometry from NASA's \kepler mission \citep{Walkowicz2013,McQuillan2013,Giles2017}.  

% We can study stellar magnetic activity like this.
Detecting the sizes and temperatures of starspots is difficult. For main sequence stars, the angular resolution required for a telescope to spatially resolve starspots is beyond the diffraction limit in the optical and infrared. Clever analysis of stellar spectral absorption features can reveal the positions of spots through Doppler imaging \citep{Vogt1983,Barnes2001,Strassmeier2002}, or its magnetic equivalent Zeeman Doppler imaging \citep{Donati2003,Morin2008,Morin2010,Morin2011,Morin2013}. Alternatively, one can measure the spot temperatures and covering fractions from high resolution spectra of active stars with molecular band modeling. The observed spectrum of a star is modeled as a linear combination of model or template stellar atmospheres with a hot and cooler component \citep{Neff1995,oneal1996,oneal1998,ONeal2004}. 

Another window into stellar magnetic activity was opened by \kepler transit photometry. During a transit event, when a planet occults a dark starspot, less light is missing than during occultations of brighter regions, creating a positive residual signals in transit light curves. One planet has occulted more spots than any other known to date, HAT-P-11 b \citep{Bakos2010,Winn2010,Deming2011,Sanchis-Ojeda2011,Hirano2011}. 

% Stellar activity affects exoplanet characterization in the following ways
Starspots directly affect our ability to make precise measurements of exoplanet properties. Their effect on the stellar flux adds signal of similar amplitude to the transit of an Earth-like planet for some Sun-like stars, making precise measurements of the transit depths (with photometry or transmission spectroscopy) and transit times (for TTVs) more challenging. To make matters worse, spots are cooler than the mean photosphere, introducing a wavelength-dependence to the variability in addition to time-dependence. Careful analyses of exoplanet properties must account for the effects of starspots in order to make reliable inferences about exoplanets.


