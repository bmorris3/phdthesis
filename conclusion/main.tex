
\begin{verse}
{\it When I heard the learn'd astronomer,\\
When the proofs, the figures, were ranged in columns before me,\\
When I was shown the charts and diagrams, to add, divide, and measure them,\\
When I sitting heard the astronomer where he lectured with much applause in the lecture-room,\\
How soon unaccountable I became tired and sick,\\
Till rising and gliding out I wander'd off by myself,\\
In the mystical moist night-air, and from time to time,\\
Look'd up in perfect silence at the stars.\it}
\hfill --- Walt Whitman
\end{verse}

\chapter{Conclusions}

\section{Summary}

In Part~\ref{part:solaractivity} we studied the magnetic activity of the Sun, and its impact on observations of Earth-like exoplanets orbiting Sun-like stars. In Chapter~\ref{chap:solarbenchmark} we reconstructed the rotational modulation of the Sun�over 70 years of observations using archival sunspot records. In Chapter~\ref{chap:plato} we found that stellar oscillations and granulation are a significant source of astrophysical noise in measuring exoplanet radii with precision, with applications to observations by the PLATO mission. In Chapter~\ref{chap:astero} we exploit a numerical technique for efficiently generating long-term light curves of Sun-like oscillators including the effects of $p$-mode oscillations, granulation and supergranulation. In conclusion, Part~\ref{part:solaractivity} built a foundational understanding of solar magnetic activity and variability, setting the stage for studies of distant stellar surfaces.

In Part~\ref{part:activity} we presented detailed studies of the magnetic activity of HAT-P-11, a K4V with a transiting hot-Neptune in a fortuitous orbit. The planet transits a chord across the star from near the south rotational pole of the star to the north pole, which allowed us to resolve starspots as a function of latitude on the stellar surface in Chapter~\ref{chap:h11}. The spots reveal that the star likely has a Sun-like dynamo, producing spots with a similar latitudinal distribution, though the spot coverage is two orders of magnitude larger. In Chapter~\ref{chap:h11sindex} we studied the long-term chromospheric activity of HAT-P-11 and found that the star likely has a $\gtrsim11$ year activity cycle similar to the Sun's, and we provide evidence that the close-in planet may be inducing some stellar activity. 

In Chapter~\ref{chap:nephelion} we investigated whether plages regions are spatially associated with dark starspots on a young G star and two K stars, and find evidence for a Sun-like association between starspots and plages for the G star, but no such evidence for the K stars, which seem to have global chromospheric activity not closely associated with the dark starspots driving rotational modulation. We measured the stellar temperatures, surface gravities, and spot covering fractions for a sample of bright stars in Chapter~\ref{chap:freckles}, and find that chromospheric activity weakly correlates with spot covering fraction, and that high precision color priors are crucial to measuring sensible spot coverages. 

We turned our attention to the magnetic activity of the very lowest mass planet-hosting star, TRAPPIST-1, in Chapter~\ref{chap:trappist1_bright}. Using \kepler and \spitzer photometry, we reasoned that TRAPPIST-1 likely has bright regions driving the stellar variability observed at optical wavelengths but not observed in the near infrared. We also found evidence that the presence of the bright regions seems to be associated with flare occurrence, perhaps indicating that the apparent rotational modulation observed at optical wavelengths may not be rotational modulation after all. Finally, we looked to astrometry for a novel measurement of extreme stellar magnetic activity by its influence on stellar astrometric centroids in Chapter~\ref{chap:gaia}.  

%In Part~\ref{part:planets} we studied a few transiting exoplanets using transit photometry to measure timing variations for TRAPPIST-1 b and c, and several targets as part of the KOINet global transit timing follow-up program, demonstrating high precision photometry with the ARC 3.5 m Telescope at Apache Point Observatory. We also observed starspot occultations in the HAT-P-11 system using a holographic diffuser on the ARC 3.5 m Telescope, further enhancing our ability to study planets and their host stars from the ground. 

Finally, in Part~\ref{part:synthesis} we brought together our knowledge of stellar magnetic activity and exoplanet characterization to prepare for observations of planets orbiting active host stars. In Chapter~\ref{chapter:robin} we consider a ``worst-case scenario'' where starspots affect our measurements of exoplanet radii, and found that robust exoplanet radii can still be recovered using the durations of ingress and egress, independent of the transit depth. In Chapter~\ref{chapter:robin_trap} we apply this technique to new \spitzer observations of TRAPPIST-1 and find no evidence for significant contamination of the transit chords of the TRAPPIST-1 planets by stellar activity. Lastly, in Chapter~\ref{chapter:jwst_nirspc} we prepare for observations of the systems of potentially habitable planets of TRAPPIST-1, Kepler-62 and Kepler-296 with the James Webb Space Telescope (JWST) using NIRSpec's in Prism mode. We determined that TRAPPIST-1 may be one of the best known systems for follow-up with James Webb due largely to its brightness and the small size of its host star. We outlined a plan for measuring the mass of the potentially habitable planet Kepler-62 f with�precision as good as 10\%, and showed that transmission spectroscopy of the TRAPPIST-1 planets can discriminate between different climates in the absence of clouds. 

\section{Future Work}

\subsection{Outlook for exoplanet detection with TESS and ground-based follow-up}

The Transiting Exoplanet Survey Satellite (TESS) will observe 200,000 stars at 2 minute cadence, providing a \kepler-like photometric database for the brightest stars distributed throughout the entire sky. Unlike \kepler, most TESS stars will be observed for only 27 consecutive days, compared to the four years of \kepler photometry. As a result of these short observing windows, when it comes to characterizing exoplanet host star magnetic activity, we will have to make more with less space-based photometry, and lean more heavily on ground-based follow-up observations. For TESS targets to be amenable to follow-up characterization with JWST, they must have either little or well-characterized stellar activity. I am prepared to provide a multi-pronged approach to characterizing the activity of exoplanet host stars observed with TESS, in preparation for characterization by JWST. I will achieve this goal with TESS photometry, and ground-based follow-up observations.

Holographic diffusers enable high precision photometry from the ground \citep{Stefansson2017,Stefansson2018} -- diffuser photometry may be a key asset for follow-up of TESS targets. The diffuser transforms the point-spread function of a star from a nearly Gaussian distribution with a tight radial spread of only a few pixels to a broad top-hat shaped distribution covering an area of thousands of pixels. For bright stars, spreading the light out over many pixels is advantageous for several reasons: (1) it enables longer exposures, which can become long enough to average over atmospheric scintillation noise; (2) it makes the photometry less sensitive to fine pointing errors and intra-pixel sensitivity variations, which are significant when most of the stellar flux is landing on only a handful of pixels; and (3) the photometry is also less sensitive to inter-pixel sensitivity variations (or flat-field precision), since the flux is spread roughly evenly over many pixels. This last point is one of the key advantages in favor of diffuser photometry over defocused photometry, where the flux is often concentrated unevenly onto a few pixels.

\subsubsection{Stellar activity of TESS host stars}

The clearest window into small-scale magnetic activity of exoplanet host stars to date is spot occultation mapping, which reveals the positions and sizes of starspots as exoplanets occult them during transits \citep{Sanchis-Ojeda2011}. When coupled with Rossiter-McLaughlin observations which constrain the spin-orbit alignment of an exoplanet system, spot occultations can reveal the latitudes and longitudes of spots on the stellar surface, allowing for direct comparison with solar activity \citep{Winn2010, Morris2017a}. I will search the TESS photometry of exoplanet host stars for every spot occultation observed during by TESS mission. For a subset of the stars with Rossiter-McLaughlin observations, I will reconstruct the latitude/longitude distributions of the spots, enabling precision comparisons between stellar and solar activity, as I did in \citep{Morris2017a}. 

Since TESS will observe many more late K and M type stars than \kepler \citep{Muirhead2018}, the mission will enable us to construct spot occultation maps for more stars across the low-mass end of the main sequence, allowing us to probe the behavior of stellar dynamos as a function of stellar mass. At present, small-scale spot maps are only available for a handful of stars, such as Kepler-17, Kepler-63, HAT-P-11 and the Sun \citep{Solanki2003,Davenportthesis,Sanchis-Ojeda2013,Morris2017a}. TESS may open a window into the dynamos of low-mass stars, allowing us to determine how magnetic activity changes with rotation period, tachocline depth, Rossby number, or other fundamental stellar parameters which are thought to influence stellar dynamos \citep{Berdyugina2005, Gilman2018}. 

Diffuser photometry is not only useful for measuring transits and and transit timing variations, but the uniquely high precision diffuser photometry is well-suited to measuring spot occultations from the ground \citep{Morris2018d}. Should more stars be discovered like HAT-P-11 which show frequent spot occultations, ground-based diffuser photometry may allow us to measure properties stellar magnetically active regions over very long baselines, for example, throughout stellar activity cycles. Also exciting is the possibility of expanding upon the monochromatic photometry of TESS by observing spot occultations from the ground in a different bandpass. For example, we can collect ground-based photometry of HAT-P-11 in the SDSS $u$ band while TESS observes in its redder bandpass, allowing us to use the simultaneous observations to roughly estimate the color, and therefore the temperature, of starspots on HAT-P-11 (or other stars). 

\subsection{Outlook for exoplanet characterization with JWST}

The TESS planet yield, in addition to ground-based surveys like SPECULOOS \citep{Delrez2018b} and SAINT-Ex, should produce a handful of planets which are ideal for characterization with the James Webb Space Telescope (JWST). JWST may provide some of the clearest views into exoplanet atmospheres and their masses to date using transmission spectroscopy and transit timing variations (TTVs), as we discussed in Chapter~\ref{chapter:jwst_nirspc}. 

While stellar magnetic activity, granulation and oscillations will enforce fundamental noise floors on the precision of JWST observations of exoplanets, we are optimistic that some targets may be bright and quiet enough to measure the masses of potentially habitable (cool) worlds via TTVs, and to measure the presence or lack of an atmosphere for some close-in (hot) rocky planets. 

If we are lucky enough to have a system where both techniques can be applied to the same planet, in principle, we can measure the bulk density of the planet, {\it and} the atomic and molecular constituents of its atmosphere. For the first time, we may be able to say something about the bulk chemistry of an exoplanet's interior as well as basic properties of its atmosphere like the presence or lack of clouds. In the case of a cloud-free atmosphere, we might also be able to discern between different climates on the planet via its atmospheric chemistry. 

\subsubsection{Outlook for astrobiology}

The aforementioned observations are the primary ingredients that go into determining the true habitability of an exoplanet. It is possible that JWST will usher us into a new era in the study of exoplanet habitability, leaving behind the rough definition of habitability based on the instellation, by using the moist and maximum greenhouse effects as boundaries of the habitable zone, and moving onward to identifying whether or not the observed surface and climate properties of a�specific planet are consistent with a habitable planet or not. With careful consideration given to the properties of the host stars, we are poised to ask one of the paramount questions of astrobiology: where are the conditions for life met in the�local Universe?
