%\documentclass[fleqn,usenatbib]{mnras}
%
%\usepackage{newtxtext,newtxmath}
%\usepackage[T1]{fontenc}
%\usepackage{ae,aecompl}
%
%\usepackage{natbib}
%%\bibliographystyle{humannat}
%
%\usepackage{graphicx}	% Including figure files
%\usepackage{amsmath}
%\usepackage{hyperref}
%\usepackage{listings}
%\usepackage{color}
%\usepackage{float}
%\usepackage{xspace} 
%\usepackage{longtable}
%
%\newcommand{\kepler}{\textsl{Kepler}\xspace}
%
%\title[Stellar Variability in Exoplanet Transits with PLATO]{The Stellar Variability Noise Floor for Transiting Exoplanet Photometry with PLATO}
%
%\author[Morris et al.]{Brett M. Morris,$^{1}$\thanks{E-mail: morrisbrettm@gmail.com}
%Monica G. Bobra,$^{2}$ Eric Agol$^{1}$, Yu Jin Lee,$^{3}$ Suzanne L. Hawley$^{1}$\\
%% List of institutions
%$^{1}$Astronomy Department, University of Washington, Seattle, WA 98195, USA\\
%$^{2}$W. W. Hansen Experimental Physics Laboratory, Stanford University, Stanford, CA 94305, USA\\
%$^{3}$Stanford University, Stanford, CA 94305, USA\\
%}
%
%% These dates will be filled out by the publisher
%\date{Accepted XXX. Received YYY; in original form ZZZ}
%
%% Enter the current year, for the copyright statements etc.
%\pubyear{2018}
%
%% Don't change these lines
%\begin{document}
%\label{firstpage}
%\pagerange{\pageref{firstpage}--\pageref{lastpage}}
%\maketitle
%
%\begin{abstract}
%One of the main science motivations for the ESA PLAnetary Transit and Oscillations (PLATO) mission is to measure exoplanet transit radii with 3\% precision. In addition to flares and starspots, stellar oscillations and granulation will enforce fundamental noise floors for transiting exoplanet radius measurements. We simulate light curves of Earth-sized exoplanets transiting continuum intensity images of the Sun taken by the Helioseismic and Magnetic Imager (HMI) instrument aboard the Solar Dynamics Observatory (SDO) to investigate the uncertainties introduced on the exoplanet radius measurements by stellar granulation and oscillations. After modeling the solar variability with a Gaussian process, we find that the amplitude of solar oscillations and granulation is of order 100 ppm -- similar to the depth of an Earth transit -- introducing radius uncertainties near 4\% over the mission duration of four years. We find that surface brightness inhomogeneity due to photospheric granulation contributes a lower limit of only 2 ppm to the photometry in-transit. The radius uncertainty due to granulation and oscillations accounts for a significant fraction of the error budget of the PLATO mission, before detector or observational noise is introduced to the light curve.
%\end{abstract}
%
%\begin{keywords}
%Sun: granulation, photosphere; planets and satellites: fundamental parameters; stars: solar-type
%\end{keywords}
%
\section{Introduction}
 
Precision space-based photometry of transiting exoplanets contains information about the host star as well as the exoplanet. Light curves of transiting exoplanets have been used to map stellar surface brightness variations within the transit chord, occasionally revealing maps of starspots \citep[see e.g.][]{Desert2011, Sanchis-Ojeda2011, Bonomo2012, Davenport2015thesis, Morris2017a, Morris2018b}. As a result, one must be careful to account for stellar surface inhomogeneities when measuring the radius of an exoplanet \citep{robin}. 

The Sun and Sun-like stars exhibit photospheric granulation, a small-amplitude surface inhomogeneity observable in transit photometry \citep{Chiavassa2017}. Stellar granulation is the pattern of bright upwelling convective plasma surrounded by dimmer down-flowing plasma which tiles the photosphere of a Sun-like star. Granulation occurs on a variety of size scales. The most prominent in typical continuum images of the Sun are granules with radii $\sim 1$ megameter (Mm) in size (see Figure~\ref{fig:hmi}), or roughly $0.2 R_\oplus$; less obvious is the structure of supergranules, which span $\sim 16$ Mm in radius, or roughly $2.5R_\oplus$ \citep{Rast2003}. 

The degree to which granules and supergranules affect transit light curves depends on their size scales. Several granules are occulted by an Earth-sized planet -- 6.37 Mm in radius -- at any instant during an exoplanet transit event. However, only a portion of a supergranule is occulted during an exoplanet transit. As a result, transits of Earth-sized exoplanets will reveal the pattern of stellar supergranulation in the residuals of extremely high precision transit photometry. 

The Sun and Sun-like stars also exhibit $p$-mode oscillations \citep{Christensen-Dalsgaard2002}. Spatially-resolved Doppler velocity and photometric measurements of the Sun reveal $p$-mode oscillations with a maximum power near period $P=5.39 \pm 0.05$ minutes \citep{Fröhlich1997, Huber2011}, often referred to as the solar ``five-minute oscillations''. These oscillations are omnipresent in solar and stellar photometry, injecting a continuous source of astrophysical noise into transiting exoplanet light curves.

The ESA PLAnetary Transits and Oscillations (PLATO) mission aims to discover transiting exoplanets among 15,000 dwarf and sub giant stars with $V<11$ with 25 second cadence photometry collected over 4.5 years \citep{Rauer2014}. One of the main scientific motivations for PLATO is to measure Earth-like exoplanet transit radii with 3\% precision. To achieve this ambitious goal for planets orbiting Sun-like stars, the host stars likely must have little stellar activity in the form of starspots or flares, and the stellar oscillations must be accounted for in the transit analysis. Sun-like stars will likely exhibit $p$-mode oscillations, which impose a noise floor on transit depths -- and therefore planet radii -- measured from photometry. 

The continuum intensity image time series from the Helioseismic and Magnetic Imager (HMI; \citealt{schou12}) aboard the Solar Dynamics Observatory (SDO) provides us with an ideal dataset for simulating transits of exoplanets on observations of the Sun. We discuss the HMI instrument in Section~\ref{sec:hmi}. In this work, we artificially superimpose an exoplanet on HMI continuum intensity images to create simulated transits of the real Sun. We then examine these transits, which contain the signatures of solar granulation and $p$-mode oscillations, and, in Sections~\ref{sec:sdohmi} and \ref{sec:photometry}, quantify the impact of each component on the exoplanet radius uncertainties. We discuss the results in Section~\ref{sec:discussion} and summarize in Section~\ref{sec:conclusion}.

\section{The HMI Instrument} 
\label{sec:hmi}
The HMI instrument aboard SDO takes a series of images in 6 wavelengths, centered on the Fe I spectral line at 6173 \AA, and 4 polarizations, which are then processed together to derive physical observables \citep{schou12}. One observable is the continuum intensity (see Figure~\ref{fig:hmi}), which is derived by reconstructing the solar line from the Doppler-shift, line-width, and line-depth estimates following Equation 14 of \citet{couvidat16}, who show that the continuum intensity derived in this manner differs from the true continuum intensity by less than 1\%.

Photon noise accounts for approximately 0.01\% of these continuum intensity measurements \citep{couvidat16}; similarly, systematic variations in the flat field on the order of 100 ppm add an additional source of noise to the data. These numbers exceed the performance specifications for the instrument.

As with any telescope, optical limitations also affect the image quality. Ideally, the light from a singular point on the Sun corresponds to a singular point on an image of the Sun. In practice, diffraction and imperfect optics cause light from a singular point on the Sun to spread over many points on an image of the Sun \citep{couvidat16}. This spread is characterized by a Point Spread Function (PSF). To mitigate the effects of diffraction and imperfect optics, we deconvolve HMI continuum intensity images of the Sun with the PSF (modeled as an Airy function; for details, see \citealt{wachter12}). This process increases the granular contrast by a factor of two (see Figure 27 of \citealp{couvidat16}), and ensures that optical artifacts play a small role in the following simulated transits. 

We also remove the diurnal variation in the amplitude of the HMI continuum intensity measurements, which appear due to the orbital velocity of SDO spacecraft (see Section 2.10 of \citealp{couvidat16}), that would otherwise dominate over the small-scale perturbations we are trying to measure in this work. We discuss our methods for removing this signal in Sections~\ref{sec:sdohmi} and \ref{sec:photometry}.

All the HMI data are publicly available at \url{http://jsoc.stanford.edu}.

\begin{figure}
    \centering
    \includegraphics[scale=0.3]{stash/continuum_small.png}
    \caption{Deconvolved SDO/HMI continuum intensity image of the quiet Sun on 21 January 2018 at 19:34 TAI shown in the Helioprojective Cartesian coordinate system with the color table scaled from 0 to 65000 DN/s.}
    \label{fig:hmi}
\end{figure}

\begin{figure}
    \centering
    \includegraphics[scale=2.475]{stash/continuum_partial.png}
    \caption{Same as Figure \ref{fig:hmi}, but zoomed in on a 128 x 128 arcsecond patch of the the quiet Sun at disk center. The black circle shows the scale of a simulated Earth-sized exoplanet.}
    \label{fig:hmi2}
\end{figure}

\section{Flux Variations of the Regions Occulted by the Planet} \label{sec:sdohmi}

\subsection{Simulating transits}

\begin{figure*}
    \centering
    \includegraphics[scale=0.9]{stash/lc.pdf}
    \caption{282 simulated transit light curves of an Earth-sized planet transiting the SDO/HMI images of the Sun (left: full transit, right: zoom into mid-transit). The spread at flux minimum is due to solar surface brightness variations due to activity and granulation.}
    \label{fig:lc}
\end{figure*}

To simulate transits, we first select a single, static continuum intensity image. We then create a time series of fluxes by first creating a synthetic exoplanet, equal to the projected size of the Earth, and summing the flux within this circular disk. We then subtract this sum from the total solar disk intensity. The projected size of the Earth is computed from the HMI image headers, which record the apparent solar radius, in units of pixels, for each image. We advance the planet incrementally across the solar surface and compute the time of each integration assuming a circular orbit with $P_\mathrm{orb} = 365$ days. We create this time series across a single, static image to avoid the diurnal variation in the amplitude of the intensity measurements. This has the added benefit of eliminating the effects of photon noise (since the entire time series is across one realization of photon noise), which allows us to probe the effects of photospheric granulation at a ppm level. 

We simulate transits across 282 deconvolved continuum intensity images taken on days in 2018 that show the least magnetic activity. This ensemble of transits is shown in Figure~\ref{fig:lc}. We only simulate transits with an impact parameter of $b=0$. Selecting $b=0$ ensures that the planet is unlikely to occult magnetically active regions, which rarely occur near the solar equator, and maximizes the granulation contrast. The software for these simulations is available online\footnote{Open source software: \url{http://github.com/bmorris3/stash}}. 

\subsection{Granulation and supergranulation noise in the transit residuals} \label{sec:granulation}

We fit the \citet{Mandel2002} transit model with quadratic limb-darkening to each of the simulated light curves, implemented by the Python package \texttt{batman} \citep{Kreidberg2015}. For each light curve, we simultaneously fit for the planet radius, mid-transit time, orbital inclination and four-parameter nonlinear limb-darkening. We then examine the residuals of each transit fit (an example is shown in Figure~\ref{fig:example_residuals}) and reject any transits contaminated by magnetic elements (transits containing residual flux values $>5$ ppm). To remove imperfections in the HMI deconvolved image flat field, which we discussed in Section~\ref{sec:hmi}, we subtract the residual flux at each time by the median of all transit residuals. The resulting transit residuals typically have standard deviations of 0.5 ppm, and a typical range is $2-4$ ppm.

\begin{figure*}
    \centering
    \includegraphics[scale=0.6]{stash/example_residuals.pdf}
    \caption{Transit residuals of a particular realization of the solar granulation, showing residual amplitude $\sim 2$ ppm. }
    \label{fig:example_residuals}
\end{figure*}

\subsection{Radius uncertainty due to granulation and supergranulation} \label{sec:gp}

Granulation occurs on a variety of size scales -- granules with radii of $\sim 1$ Mm, which are clearly visible in the HMI continuum intensity images (see Figure~\ref{fig:hmi}), and supergranules, with radii of $\sim 16$ Mm, which are not easy to see in the continuum intensity images \citep{Nordlund2009}. Granules and supergranules appear stochastically over time, but their length scales and turnover times stay roughly constant. Therefore, we can model our ensemble of 282 transit residuals as an autocorrelated signal with noise. 

We model the autocorrelated signal with Gaussian process regression -- for more information on Gaussian processes, see \citet{Rasmussen2006}. We use the following simple harmonic oscillator kernel implemented by the Python package \texttt{celerite} \citep{Foreman-Mackey2017}:
\begin{equation}
S(\omega) = \sqrt{\frac{2}{\pi}} \frac{s_0 \omega_0^{4}}{(\omega^2 - \omega_0^2)^2 + \omega_0^2\omega^{2}/Q^2}.
\label{eqn:sho}
\end{equation}
We use a Markov Chain Monte Carlo method, implemented by the Python package \texttt{emcee} \citep{Foreman-Mackey2013}, to simultaneously fit for the simple harmonic oscillator kernel hyperparameters $S_0$ and $\omega_0$ (while holding $Q$ constant) as well as the \citet{Mandel2002} transit model parameters for the orbital inclination $i_o$, quadratic limb darkening parameters $u_1$ and $u_2$, mid-transit time $t_0$, and $R_p/R_\star$, the ratio of the exoplanet radius, $R_p$, to the stellar radius, $R_\star$. 

We find a typical uncertainty on the exoplanet radius is 0.02\% $R_p$. We consider this a noise floor, or a lower limit, since we modelled the granulation pattern from a synthetic transit across a single, static image. Modelling the granulation across a time-varying background would account for the convective motions of the granules and therefore introduce a more complex pattern for the granulation noise.

We also find that the transit residuals are roughly periodic with timescale $P = 18 \pm 1$ minutes. We can use this periodicity to measure the characteristic length scale for supergranulation. Assuming a supergranule is large compared to the planet, the duration of a supergranule occultation $\tau$ by a small transiting exoplanet (with an impact parameter $b=0$) is approximately

\begin{equation}
\tau \approx \frac{2R_\mathrm{sg}}{v} = \frac{R_\mathrm{sg} P_\textrm{orb}}{\pi a} \label{eqn:tau}
\end{equation}

where $R_\mathrm{sg}$ is the radius of the supergranule, $v$ is the orbital velocity of the planet, $P_\textrm{orb}$ is the orbital period of the planet, and $a$ is the semimajor axis of the planet's orbit. Rearranging Equation~\ref{eqn:tau} for the supergranule radius, we find $R_\mathrm{sg} = 16 \pm 1$ Mm, similar in horizontal scale to supergranules observed both in radial velocity maps of the solar surface \citep{Hathaway2000} as well as in continuum intensity \citep[see for example][]{Meunier2008, Goldbaum2009, Rieutord2010}. Supergranulation imprints itself in patterns on the Sun with characteristic lifetimes of $\sim 20$ hours \citep{Rast2003}, which is longer than a typical Earth transit of a Sun-like star ($\sim 12$ hours).

\subsubsection{Comparison with previous results}

\citet{Chiavassa2017} simulated transits of Earth-like planets on three-dimensional radiative hydrodynamical simulations from the {\sc Stagger} grid models of Sun-like stars and found the residual signal in the transit due to granulation had RMS amplitude 3.5 ppm in the bandpass 7600-7700 \AA, which is similar to our estimate using HMI continuum intensity images ($2-4$ ppm).

\subsubsection{Expectations for other stars}

Numerical simulations of stellar granulation for stars across the main sequence show that granule size scales inversely with the stellar surface gravity \citep[see review by][]{Nordlund2009, Kupka2017}. As a result, one might expect that stars smaller than the Sun will have smaller granules. The smaller the granules, the more granules occulted by an Earth-sized planet in a given exposure, and therefore the smaller the in-transit signal of granulation on the light curve. 

Numerical simulations of granulation for stars from spectral type F7-K3 dwarfs by \citet{Trampedach2013} all have characteristic horizontal scales of granules of order 1 Mm. These scales grow as stars evolve and their $\log g$ decreases \citep[see also][]{Trampedach2017}. Therefore the small-scale granulation signal should be most significant for evolved stars.

Solar supergranulation, in contrast with small scale granulation, is difficult to measure due to its small amplitude, and difficult to simulate due to its vast physical extent \citep{Rieutord2010}. Thus we caution the reader to only use the results of this analysis for Sun-like stars.

\section{Flux Variations of the Unocculted Stellar Disk} 
\label{sec:photometry}

\subsection{Simulating transits}

\begin{figure*}
    \centering
    \includegraphics[scale=0.6]{stash/oscillations.pdf}
    \caption{24 hours of 45 second cadence photometry of the Sun from SDO/HMI continuum intensity images, with the mean intensity removed to show the small amplitude variability of the Sun on these timescales (black circles). The error bars on the fluxes from each measurement are smaller than the points. The blue curve is the maxmimum-likelihood Gaussian process fit to the light curve with a simple harmonic oscillator kernel. }
    \label{fig:pmodephot}
\end{figure*}

In Section~\ref{sec:sdohmi} we studied the effect of stellar brightness variations due to granulation within the transit chord of an Earth-like exoplanet on the transit light curve. We assumed that the disk-integrated brightness of the star was unchanging, and only the spatial surface brightness variations within the transit chord were responsible for perturbations to the transit residuals. While this was a pedagogically interesting exercise, it was fundamentally one step removed from real photometry of stars -- $p$-mode oscillations, magnetic activity and granulation inject time variability into the disk-integrated flux of a Sun-like star. In this Section, we measure the amplitude of the out-of-transit variability of the star.

First, we measure the total brightness of the Sun using 45-second cadence HMI continuum intensity images throughout four sets of 24-hour observations, each separated by one year. The four sets of observations represent one transit observation for each of the four years of the nominal PLATO mission. This disk-integrated variability will affect both in- and out-of-transit photometry observed by PLATO. We compute the total intensity of each HMI image taken on 21 January 2018, a day with little magnetic activity, as well as the same day in 2015, 2016, and 2017. 

The resulting photometry has a strong diurnal signal, due to the orbital velocity of the SDO spacecraft (see Section 2.10 of \citealp{couvidat16}), which dominates over the small-scale perturbations we are trying to measure in this work. Our second step is to remove this trend by modelling the light curve with a smooth Gaussian process using a Mat{\'e}rn 3/2 kernel \citep{Rasmussen2006}, and divide the light curve by the maximum-likelihood fit.

The resultant systematics-corrected HMI photometry for 21 January 2018 is shown in Figure~\ref{fig:pmodephot} (black circles), with the mean flux from the time series removed. The amplitude of the variability in the 45-second cadence photometry is $\sim$100 ppm. We also show the maximum-likelihood Gaussian process fit with a simple harmonic oscillator kernel (blue curve). 

\begin{figure*}
    \centering
    \includegraphics[scale=0.75]{stash/transits.pdf}
    \caption{Transits of an Earth-sized planet across the solar surface, including the disk-integrated variability due to $p$-mode oscillations and granulation. Clearly the disk-integrated variability is greater than the in-transit granulation signal discussed in Section~\ref{sec:sdohmi} (1 ppm).}
    \label{fig:transitphotometry}
\end{figure*}

Next we inject exoplanet transits with Earth's size and orbit (with $e=0$), shown in Figure~\ref{fig:transitphotometry}, into the four 24-hour sets of systematics-corrected photometric time series measurements. To do this, we could superimpose a synthetic planet on each image, as we did for a single, static image in Section~\ref{sec:sdohmi}, compute the total intensity per image, and remove the diurnal signal by modelling the light curve with a smooth Gaussian process using a Mat{\'e}rn 3/2 kernel. However, this approach may remove the small-scale perturbations we are trying to measure.

Instead, we use a model to construct a synthetic \citep{Mandel2002} transit model and we multiply this with each of the four 24-hour sets of systematics-corrected photometric time series measurements as observed by HMI. The model neglects the effects of foreshortening; however, since the radial component of $p$-mode oscillations is much greater than its transverse component, the strongest $p$-mode oscillation signals occur near disk center.

These simulated transits are represent what PLATO will observe if it discovers a true Earth-analog orbiting a Sun-like star. The scatter in the plots is due to solar oscillations and granulation. The amplitude of the oscillations is similar to the transit depth of the Earth. In the HMI photometry, the error bars are similar in scale to the size of the points in the plot. 
 
\subsection{Radius uncertainty due to oscillations} \label{sec:gp2}

\begin{figure}
    \centering
    \includegraphics[scale=0.8]{stash/radius_posterior.pdf}
    \caption{Posterior distribution of the radius of the Earth-sized planet, in units of Earth radii. The uncertainty is 4\%, accounting for the full noise budget of the PLATO mission, before any instrumental or systematic effects are accounted for. The long tail towards larger radii is due to degeneracy with impact parameter (see Appendix~\ref{sec:posteriors} for full posterior distributions).}
    \label{fig:radius_posterior}
\end{figure}

To determine the uncertainty in an exoplanet's radius due to solar oscillations and granulation, we follow the same procedure as in Section~\ref{sec:gp}. We simultaneously fit a \citet{Mandel2002} transit light curve and a Gaussian process with a simple harmonic oscillator kernel to the transit photometry in Figure~\ref{fig:transitphotometry} to measure the uncertainty in the exoplanet radius while accounting for the correlated noise due to oscillations and granulation. In the previous section, the periodicity measured by the kernel arose from the synthetic planet spatially crossing bright convective upflows and dark inter-granular lanes. In this case, the periodicity measured by the kernel is due to $p$-mode oscillations and granulation.

The posterior distribution for the planet radius is shown in Figure~\ref{fig:radius_posterior}. The uncertainty in the planet radius is 3.2\%. This uncertainty is the effect of three  contributions: the degeneracy with impact parameter, and the signals of stellar granulation and pulsations. This predicted uncertainty amounts to the full error budget in the mission specifications for PLATO. Marginal improvements on the precision of the transit light curves may be obtained by modeling more of the out-of-transit light curve, or with an extended mission which observes the same field for more transits. 

The oscillation period of the maximum-likelihood Gaussian process is $P=5.83 \pm 0.10$ minutes (see Appendix~\ref{sec:posteriors} for full posterior distributions). This is slightly longer than the canonical ``five-minute oscillation'' period of $5.39 \pm 0.05$  min \citep{Fröhlich1997,Huber2011}.

\section{Discussion} \label{sec:discussion}

The PLATO mission seeks to measure Earth-analog exoplanet radii with 3\% precision. We have demonstrated, using simulated photometry from HMI continuum intensity images, that stellar oscillations and granulation, in addition to degeneracy between the planet radius and orbital impact parameter will be important contributors to the uncertainty in exoplanet radii, of order 3\%. In order to reach 3\% precision, we took advantage of the extremely high signal-to-noise of the HMI observations and fit a Gaussian process to the solar oscillations and granulation.

The HMI photometry detrending strategy in Section~\ref{sec:photometry} used to remove the continuum intensity trends with orbital phase likely also removed solar variability on timescales greater than a few hours. As a result, there may be additional sources of correlated noise that are not incorporated into the radius uncertainties we report in this work. Primarily, we note that super- and meso-granulation are sources of correlated noise on the hours-to-days timescales similar to the transit of an Earth across a Sun \citep{Fröhlich1997, Aigrain2004}. As such, the 4\% radius uncertainty noise floor that we present here should be understood as a lower limit -- incorporating the longer timescale sources of variability will increase the radius uncertainty. 

We test whether the detrending technique is removing significant signals that will affect the exoplanet radius uncertainty by downloading four complete days of continuous, 1-minute cadence continuum intensity images taken by the Michelson Doppler Imager (MDI; \citealt{scherrer95}) aboard the Solar and Heliospheric Observatory (SoHO). MDI took data from 1996 until 2010 from the L1 Lagrange point, and thus its continuum intensity data does not show the orbital phase variations present in the SDO data. Therefore, the MDI observations\footnote{MDI observations are publicly available at \url{http://jsoc.stanford.edu}} are an ideal control data set for comparison against the detrended HMI continuum intensity photometry. We find that after injecting transits into the MDI observations, the posterior distributions for the exoplanet radius have the same uncertainty as the detrended HMI observations (4\%). This indicates we do not significantly underestimate the uncertainty in the exoplanet radius due to the HMI detrending process.

\subsection{Wavelength dependence of variability}

In the calculations presented above, we assumed that the variability that PLATO will observe is similar to the ``psuedo-continuum'' variability observed by SDO/HMI. However, HMI continuum intensity is observed over a narrow $\sim 1$ \AA\ bandpass centered on 617.3 nm, whereas PLATO will observe in a broad bandpass from 500-1050 nm. Therefore we must verify that the variability observed in the narrow HMI bandpass is not significantly different from the variability extrapolated into the PLATO bandpass. We can approximate the ratio of variability in the HMI bandpass to the variability in the PLATO bandpass as follows.

Ignoring limb-darkening, let's say a fraction $f_c \ll 1$ of the star is covered by the cool portion of granules with temperature $T_c$, while the remainder of the star we take as having a constant temperature, $T_s$.
Then, the flux from the star is given by:
\begin{equation}
F _ { \nu } = \Omega \left[ f _ { c } I _ { \nu } \left( T _ { c } \right) + \left( 1- f _ { c } \right) I _ { \nu } \left( T _ { s } \right) \right]
\end{equation}
we will assume the granule temperature is constant, so the only way that the flux can vary in the \kepler band is via a variation in $f_c$.  The specific intensity is $I_\nu(T)$ for an atmosphere at temperature $T$, while $\Omega$ is the solid angle of the star.

Then, integrating over the SDO/HMI (subscript $H$) and PLATO (subscript $P$) bands:
\begin{eqnarray}
\dot { N } _ { H } &=& \int d \nu \frac { F _ { \nu } } { h \nu } T _ { H ,\nu } \\ 
\dot { N } _ { P } &=& \int d \nu \frac { F _ { \nu } } { h \nu } T _ { P ,\nu }
\end{eqnarray}
where $T_{H,\nu}$ is the SDO/HMI throughput at frequency $\nu$, and $\dot{N}_H$ is the photon count rate (cm$^2$ s$^{-1}$).

When $F_\nu$ varies, this causes variation in $\dot{N}_H$ and $\dot{N}_P$, but both of these just depend on $f_c$. Since the amplitude of variation is small, we Taylor expand:
\begin{equation}
\dot { N } _ { H } = \dot { N } _ { H } | _ { f _ { 0} ,0} \left[ 1+ \left( f _ { c } - f _ { c ,0} \right) \frac { d \dot { N } _ { H } } { d f _ { c } } \frac { 1} { \dot { N } _ { H } | f _ { c ,0} } \right]
\end{equation}
Now,
\begin{multline}
\frac { d \dot { N } _ { H } } { d f _ { c } } \frac { 1} { \dot { N } _ { H } | _ { f _ { c } ,0} } = \frac { \int d \nu \left[ I _ { \nu } \left( T _ { c } \right) - I _ { \nu } \left( T _ { s } \right) \right] \nu ^ { - 1} T _ { H ,\nu } } { \int d \nu \left[ f _ { c } I _ { \nu } \left( T _ { c } \right) + \left( 1- f _ { c } \right) I _ { \nu } \left( T _ { s } \right) \right] \nu ^ { - 1} T _ { H ,\nu } } \\
\approx \frac { \int d \nu  \left[ I _ { \nu } \left( T _ { c } \right) - I _ { \nu } \left( T _ { s } \right) \right]  \nu ^ { - 1} T _ { H ,\nu } } { \int d \nu I _ { \nu } \left( T _ { s } \right) \nu ^ { - 1} T_{H ,\nu} }
\end{multline}
where the approximation assumes $f_c \ll 1$. A similar relation holds for the PLATO bandpass. Thus, the ratio of the fractional amplitude of
variation in PLATO to that in SDO/HMI, $\alpha$, is given by:
\begin{multline}
\alpha \equiv \left( \frac { d \dot{N} } { d f _ { c } } \frac { 1} { \dot { N }_P | f _ { c ,0} } 
\right)  \left( \frac { d \dot{N} _ { H } } { d f _ { c } } \frac {1} { \dot{N} _ { H } | f _ { c ,0} } \right)^{-1} \approx\\
\frac { \int d \nu [I_\nu(T_c)-I_\nu(T_s)] \nu ^ { - 1} T _ { P ,\nu } } { \int d \nu [I_\nu(T_c)-I_\nu(T_s)] \nu ^ { - 1} T _ { H ,\nu } } \frac { \int d \nu I _ { \nu } \left( T _ { s } \right) \nu ^ { - 1} T _ { H ,\nu } } { \int d \nu I _ { \nu } \left( T _ { s } \right) \nu ^ { - 1} T_{P ,\nu}}
\end{multline}
We estimate that the variability of a Sun-like star with $T_\mathrm{eff} = 5777$ K as observed in the PLATO bandpass will have 86\% of the variability observed in the SDO/HMI bandpass. We therefore assume that the variability due to granulation estimated by SDO/HMI is a good proxy for the variability that will be observed by PLATO, or slightly overestimated.

\section{Conclusion} \label{sec:conclusion}

We have constructed photometry of the Sun using HMI continuum intensity time series images to study the effects of solar $p$-mode oscillations and granulation on transit photometry. The short timescale ($<1$ hour) scatter in a transit light curve of an Earth-like planet transiting a Sun-like star has two components. First, both the in- and out-of-transit residual scatter show disk-integrated brightness fluctuations driven by $p$-mode oscillations and granulation, of order 100 ppm in amplitude with a five-minute period. 

Second, maps of stellar surface brightness variations within the transit chords of exoplanets are encoded in the residuals of transit photometry. We find that the brightness variations due to granulation impart only a slight additional scatter, of order 2-4 ppm, to the in-transit photometry, in good agreement with studies of simulated stellar atmospheres by \citet{Chiavassa2017}.

We demonstrate that transiting exoplanet radius uncertainties of 4\% are possible with photon noise-limited photometry from HMI continuum intensity images by accounting for the correlated disk-integrated $p$-mode and granulation signals with a Gaussian process. This uncertainty on Earth-like exoplanet radii accounts for the full error budget for the PLATO mission, and perhaps motivates a longer extended mission to improve upon the radius measurement uncertainties. 
%
%\section*{Acknowledgements}
%
%We are grateful to Manodeep Sinha, who helped make the numerical methods presented in this work vastly more efficient. We would also like to thank Jeneen Sommers for generating the deconvolved HMI continuum intensity data and Charles Baldner for discussions about the HMI instrument. This research has made use of NASA's Astrophysics Data System.
%We gratefully acknowledge the following software packages which made this work possible: \texttt{astropy} \citep{Astropy2013, Astropy2018}, \texttt{ipython} \citep{ipython}, \texttt{numpy} \citep{VanDerWalt2011}, \texttt{scipy} \citep{scipy},  \texttt{matplotlib} \citep{matplotlib}, \texttt{celerite} \citep{Foreman-Mackey2017}, \texttt{sunpy} \citep{sunpy}, \texttt{emcee} \citep{Foreman-Mackey2013}, \texttt{corner} \citep{Foreman-Mackey2016}. 
%
%\bibliographystyle{mnras}
%\bibliography{bibliography}
%
%\appendix

\begin{subappendices}
\section{Typical posterior distributions} \label{sec:posteriors}

In Section~\ref{sec:gp2} we injected a \citet{Mandel2002} transit model with the properties of Earth (impact parameter $b=0$) into the SDO/HMI continuum intensity photometry of the Sun, whose scatter is dominated by $p$-mode oscillations (see Figure~\ref{fig:transitphotometry}). We fit simultaneously for the transit light curve parameters $R_p/R_\star$, $i_o$, $a$, $t_0$, $u_1$ and $u_2$, in addition to the simple harmonic oscillator kernel hyperparameters $S_0$ and $\omega_0$. In Figure~\ref{fig:corner} we show the posterior distributions for each of the parameters. 

\begin{figure*}
    \centering
    \includegraphics[scale=0.38]{stash/corner_oot.pdf}
    \caption{Posterior distributions for the Gaussian process hyperparameters and the \citet{Mandel2002} transit model parameters in a typical fit to a simulated transit. The periodicity in the residuals has timescale $P = 5.83 \pm 0.10$ minutes, and the uncertainty in the planet radius is 4.8\%.}
    \label{fig:corner}
\end{figure*}
\end{subappendices}
%
%
%% Don't change these lines
%\bsp	% typesetting comment
%\label{lastpage}
%\end{document}